\documentclass[12pt]{article}

\usepackage{booktabs}
\usepackage{tabularx}

\title{Hazard Analysis\\\progname}

\author{\authname}
\date{\today}

\input{../Comments}
%% Common Parts

\newcommand{\progname}{Flick Picker} % PUT YOUR PROGRAM NAME HERE
\newcommand{\authname}{Team 7, 7eam
\\ Talha Asif - asift
\\ Jarrod Colwell - colwellj
\\ Madhi Nagarajan - nagarajm
\\ Andrew Carvalino - carvalia % AUTHOR NAMES        
\\ Ali Tabar - sahraeia
}     

\usepackage{hyperref}
    \hypersetup{colorlinks=true, linkcolor=blue, citecolor=blue, filecolor=blue,
                urlcolor=blue, unicode=false}
    \urlstyle{same}
                                


\begin{document}

\maketitle

~\newpage \pagenumbering{roman}

\tableofcontents

~\newpage

\section*{Revision History}
\begin{table}[hp]
	\caption{Revision History} \label{TblRevisionHistory}
	\begin{tabularx}{\textwidth}{llX}
		\toprule
		\textbf{Date} & \textbf{Developer(s)} & \textbf{Change}\\
		\midrule
		October 17 & Jarrod Colwell & Created document structure\\
		October 17 & Talha Asif & Modifying Doc Structure\\
		October 19 & Andrew Carvalino & Definition of Hazard and Critical Assumptions\\
		... & ... & ...\\
		\bottomrule
		\end{tabularx}
\end{table}

~\newpage \pagenumbering{arabic}

\section{Introduction}
Before going any further with system design, it is crucial to conduct a hazard analysis of the system from an engineering perspective. The goal is to identify critical safety concerns the application users could face and the solutions to them. Hazards will be determined using the Failure Modes and Effects Analysis (FMEA) for Flick Picker.

\section{Scope and Purpose}
a

\section{Background}
a

\section{System Boundary}
a

\section{Scope of Hazard Analysis}
a

\section{Definition of Hazard}
A hazard is a potential source of danger that can arise from an individual part or emergent property of a system. These dangers can be both physical as well as social and psychological, such as in the case of a person's private information being leaked to the public.

\section{Critical Assumptions}
\begin{enumerate}
	\item System will not have direct access to users' hardware (ex. specific CPU registers)
	\item Files will not be downloaded onto the users' device without the explicit consent of the user (should that be a feature of the system)
	\item Users' private information will not be sold or intentionally disclosed to any third parties
\end{enumerate}

\section{Failure Modes and Effects Analysis}
a

\section{Safety Requirements}
a

\section{Roadmap}
a

\end{document}