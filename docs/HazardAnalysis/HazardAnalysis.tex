\documentclass[12pt]{article}

\usepackage{booktabs}
\usepackage{tabularx}

\title{Hazard Analysis\\\progname}

\author{\authname}
\date{\today}

\input{../Comments}
%% Common Parts

\newcommand{\progname}{Flick Picker} % PUT YOUR PROGRAM NAME HERE
\newcommand{\authname}{Team 7, 7eam
\\ Talha Asif - asift
\\ Jarrod Colwell - colwellj
\\ Madhi Nagarajan - nagarajm
\\ Andrew Carvalino - carvalia % AUTHOR NAMES        
\\ Ali Tabar - sahraeia
}     

\usepackage{hyperref}
    \hypersetup{colorlinks=true, linkcolor=blue, citecolor=blue, filecolor=blue,
                urlcolor=blue, unicode=false}
    \urlstyle{same}
                                


\begin{document}

\maketitle

~\newpage \pagenumbering{roman}

\tableofcontents

~\newpage

\section*{Revision History}
\begin{table}[hp]
	\caption{Revision History} \label{TblRevisionHistory}
	\begin{tabularx}{\textwidth}{llX}
		\toprule
		\textbf{Date} & \textbf{Developer(s)} & \textbf{Change}\\
		\midrule
		October 17 & Jarrod Colwell & Created document structure\\
		October 17 & Talha Asif & Modifying Doc Structure\\
		October 17 & Talha Asif & Added introduction section content\\
		October 17 & Jarrod Colwell & Added scope and purpose section content\\
		\bottomrule
		\end{tabularx}
\end{table}

~\newpage \pagenumbering{arabic}

\section{Introduction}
Before going any further with system design, it is crucial to conduct a hazard analysis of the system from an engineering perspective. The goal is to identify critical safety concerns the application users could face and the solutions to them. Hazards will be determined using the Failure Modes and Effects Analysis (FMEA) for Flick Picker.

\section{Scope and Purpose}
This document covers the various areas in which the system is most vulnerable, including but not limited to:
\begin{itemize}
	\item External Resource Integration Points
	\item Server Communication
	\item TODO: Add more here or delete
\end{itemize}
Along with identifying the vulnerable areas of the system, this document also covers the strategies, both elimination and mitigation, and new security requirements to reduce or eliminate the impact that these hazards have.

\section{Background}
a

\section{System Boundary}
a

\section{Scope of Hazard Analysis}
a

\section{Definition of Hazard}
a

\section{Critical Assumptions}
a

\section{Failure Modes and Effects Analysis}
a

\section{Safety Requirements}
a

\section{Roadmap}
a

\end{document}